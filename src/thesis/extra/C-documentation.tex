
\chapter{Dokumentace použití}
\label{chapter:documentation}

V této kapitole je popsáno používání tohoto formátu a~to ať už vygenerování aplikace z~různých zdrojů tak přehrávání a~ovládání animace. 

\section{Generování animace}

Animace je generována pomocí Python aplikace, která je popsaná v~páté kapitole. Skript se pouští v~příkazové řádce počítače se vstupními parametry. Skript si sám kontroluje vstup a~poskytne v~případě problémů rychlou nápovědu. V~příkladech je použita cesta z~kořenového adresáře přiloženého CD. 

\begin{lstlisting}
$./src/impl/generator/main.py
Usage:
    main.py [ -v -o PATH -s STEP] images <path>
    main.py [ -v -o PATH] video <file>
    main.py [ -v -o PATH] gif <file>
    main.py (-h | --help)
\end{lstlisting}


Ve výpisu vidíme na prvním řádku chybu, která nastala, dále následuje nápověda, která nám ukazuje možné použití generátoru a~formát jeho vstupu. Pokud se vstup nerozpozná nebo je prázdný, program nahlásí konkrétní chybu a~ukončí se.


\subsection{Import videa}

Pokud bychom chtěli importovat video, můžeme použít jednoduchou syntaxi jako v~tomto příkladu: 


\begin{lstlisting}
generator.py video /examples/video/src/video.mp4
Running video parser.
Using default delay: 1000ms.
Found 24 frames, starting conversion.
Output written in video.mp4/ folder. 
\end{lstlisting}

Generátor projde aplikací a~úspěšně provede všechny potřebné operace. Import videa navíc dokáže přijmout další parametr, který slouží k~určení délky kroku animace. Pokud není uveden na vstupu, vybere se výchozí hodnota. V~ukázce níže je zobrazen výstup s~vlastní vstupní hodnotou.

\begin{lstlisting}
generator.py video -s 400 /examples/video/src/video.mp4
Running video parser.
Usign animation step: 400ms.
Found 48 frames, starting conversion.
Output written in video.mp4/ folder.
\end{lstlisting}


\subsection{Import sekvence obrázků}

Tento import je nejtriviálnějším importem pro generátor, jelikož jeho jediná operace je načtení obrázků a~ošetření, zda všechny obrázky existují a~fungují správně. Jeho příkladné použití je zobrazeno níže.

\begin{lstlisting}
generator.py images /examples/images/src/
Running images parser.
Found 18 frames, starting conversion.
Output written in /examples/images/src/animation/ folder.
\end{lstlisting}

\subsection{Import GIF obrázku}

Dále import podporuje animaci z~GIF obrázku. Vstupní obrázek se rozloží na jednotlivé snímky animace a~pustí se algoritmus.

\begin{lstlisting}
generator.py gif /examples/gif/src/animation.gif
Running gif parser.
Found 6 frames, starting conversion.
Output written in /examples/gif/src/animation/ folder.
\end{lstlisting}

\section{Použití animace na webové stránce}

Přehrávání animace je popsáno v~jazyce JavaScript. Počítá se, že veškeré instalační podmínky jsou splněny a~tedy je možné používát knihovnu jQuery.

\subsection{Inicializace}

Nejprve je potřeba vložit náš plugin pro přehrávání na stránku. Tento kód vložíme až po inicializaci jQuery.

\begin{lstlisting}
  <script src="html5anim.min.js" async="true"></script>
\end{lstlisting}

Na stránce je dále nutné definovat HTML element, ve kterém bude probíhat animace. Stylizace tohoto elementu je mimo obsah této práce a~v~přílohách je použit nestylizovaný HTML tag:

\begin{lstlisting}
  <div id="animation"></div>
\end{lstlisting}

Inicializace samotné animace už probíhá v~jQuery jakožto samostatný plugin. Tento plugin se stará o~veškerou funkcionalitu dané aplikace a~dokáže také posílat a~přijímat příkazy pro ovládání.

\begin{lstlisting}
  $(document).ready(function(){
    var element = $("#animation");
    // inicializace animace
    element.html5anim({
      src: 'animation/packed.png',
      timeline: 'animation/timeline.json',
      loop: false,
      autoplay: false // zacit prehravat hned po spusteni?
    });
    //konec inicializace
  });
\end{lstlisting}

Jakmile je animace načtená, je možné ji začít ovládat přes následující příkazy. Počítá se s~provedením inicializace (kódu z~předchozího odstavce).

\begin{lstlisting}
// spusteni animace (zastavi se u~konce)
element.html5anim('play');

// zastaveni animace
element.html5anim('pause');

//prehrani na 43.  krok animace
element.html5anim('playTo', 43); 

//prehrani zpet na 0. krok animace
element.html5anim('rewindTo', 0);

\end{lstlisting}

Knihovna dále nabízí možnost informovat uživatele, pokud nastane nějaká událost. Podporují se následující události:

\begin{itemize}
\item onLoad - po načtení animace 
\item onPlay - vždy po spuštění animace
\item onStop - vždy po zastavení animace (i pokud dojede do konce) 
\item onFrame ( krok animace ) - vždy při přejití na snímek

\end{itemize}

Komunikace může probíhat načtením těchto proměnných při inicializaci nebo později při komunikaci s~knihovnou.

\begin{lstlisting}
var element = $("#animation");
// inicializace animace
element.html5anim({
  src: 'animation/packed.png',
  timeline: 'animation/timeline.json',
  autoplay: false, // zacni prehravat po nacteni
  onFrame: function(i){
    //co se stane pri vykresleni i-teho snimku
    window.console.log( "Prave probiha " + i + " krok animace" );
  },
  onStop: function(){
    //co se stane pri zastaveni
    window.console.log( "Zastaveni animace" );
  }
});

//pridani odposlechu udalosti po inicializaci
element.html5anim('onPlay', function(){
  window.console.log("Animace je spustena" );
});


\end{lstlisting}