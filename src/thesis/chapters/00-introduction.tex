\begin{introduction}

Internetové technologie slouží mimo jiné jako reklamní formát, který používají firmy a~jednotlivci k~prezentaci své značky. V~dnešní době už kromě hezké grafiky lze internetovou prezentaci oživit i~animací.

Pojmem animace se myslí jakýkoliv pohyblivý obrázek, který se na stránkách nyní řeší pomocí různých technologií. Při reálném použití se vyskytují určité nevýhody. Mezi ně patří například podpora v~mobilních prohlížečích, datová velikost nebo kvalita barev u~zobrazení. 

Současné možnosti technologie HTML5 se rozšířily o~element Canvas, který slouží k~vykreslování různých objektů a~bitmap pomocí technologie Javascript. Je možné navrhnout speciální animační formát, který by využíval tyto nativní technologie.

K napsání této bakalářské práce mě inspiroval článek\cite{appleWebsite} o~stránkách firmy Apple\cite{apple}, která tento jiný formát použila\cite{engadget} při představení nového produktu.

V této práci se věnuji jednoduchým animacím, kde se na jakkoli velké ploše mění pouze část stránky bez ztráty kvality ilustrace při animaci. Jedná se například o~animaci v~logu stránky nebo záznam používání webové aplikace. V~těchto případech se nám hodí snadná ovladatelnost a~budeme klást požadavky na spuštění až po provedení nějaké události.


\section{Struktura práce}

Kapitoly práce jsou rozvrženy tak, aby odpovídaly i~skutečnému pořadí prováděných úkolů. V~první kapitole je uveden cíl práce tak, aby bylo lépe poznat, na co je třeba se v~dalších kapitolách zaměřit a~o co se pokusit v~implementaci. 

V druhé kapitole je proveden průzkum současných možných řešení a~definování možné alternativní formy pro vytvoření animace, která by stále měla prostor mezi současnou technologií.

Třetí kapitola se zabývá analýzou potřeb a~dalších požadavků pro nový animační formát. Jelikož zadáním práce je vydefinování alternativního formátu, bude potřeba vytvořit jak generátor animace, tak přehrávač.

Ve čtvrté kapitole na základě analýzy definuji formát a~popíšu algoritmus zakódování a~přehrávání. 

V dalších kapitolách se věnuji návrhu a~implementace obou částí projektu, tedy návrh a~implementace generátoru a~přehrávače ve webovém rozhraní. 

V poslední deváté kapitole jsou popsány praktická využití formátu a~zhodnocení oproti současným možnostem.


\end{introduction}
