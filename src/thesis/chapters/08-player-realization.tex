\chapter{Implementace přehrávače animace}


\section{Výběr jazyka a~programovacích knihoven}


V zadání a~požadavcích bylo stanoveno, že je potřeba dosáhnout použití v~nativních technologiích prohlížečů. Jediný skriptovací jazyk, který splňuje tuto podmínku je Javascript, takže výběr je omezený pouze na něj. Dále jsou použity některé nástroje a~knihovny, které pomáhají s~vývojem a~jsou popsány v~této sekci.

\subsubsection*{jQuery}

Pro Javascript existuje knihovna jQuery. Hlavní výhodou této knihovny je usnadnění práce s~DOM elementy a~jejich vlastnostmi mezi různými prohlížeči. Nemusíme se pak starat o~ošetření, v~jakém prohlížeči se momentálně nacházíme. Celý přehrávač bude implementován jako doplňek do tého knihovny, takže pro použití na stránkách bude potřeba mít tuto knihovnu implementovanou ve stránce. 

\subsubsection*{Grunt.js}

Při vývoji použijeme systém Grunt.js, který se bude starat o~automatizaci často opakovaných úkolů. Zejména se bude starat o~následující úkoly:


\begin{itemize}
  \item Optimalizace kódu Javascript a~vytvoření zmenšeného balíčku určeného k~nasazení 
  \item Testování aplikace
  \item Generování api dokumentace z~kódu
\end{itemize}

Upřesnění a~popis, jak se jednotlivé operace spouští, bude vysvětlen v~uživatelské příručce přehrávače \ref{chapter:documentation}.

\subsubsection*{Browserify}

Browserify je nástroj, pomocí kterého lze vytvořit jeden balíček z~více souborů. Díky tomuto nástroji je možné zdrojový kód přehledně skladovat v~několika souborech a~pomocí nástroje Grunt ho \"kompilovat\" do jednoho balíčku určeného k~použití. 



\section{Postup vývoje}

Vývoj probíhal ve dvou částech. Jakmile generátor vytvořil první validní výstup, byl vytvořen primitivní skript, který dokázal daný formát přehrát v~rámci webové stránky.

\subsection{Primitivní verze algoritmu}

Zmíněná primitivní verze měla za úkol pouze zobrazit definovaný formát a~zobrazit ho na webové stránce. Pro příklad je zde zobrazena testovací nejprimitivnější verze funkčního přehrávače.

\begin{figure}[h!]
\caption{Kód jednoduchého algoritmu pro vykreslení}
\begin{lstlisting}
<script type="text/javascript" src="anim_metadata.js"></script>
<canvas id="canvas" width="354" height="354"></canvas>
<script>
  var img = new Image();
  var _timeline = $.getJSON('anim_metadata.json')
  var ctx = document.getElementById("canvas").getContext('2d');
  img.onload = function()
  {
    var i = 0; // aktualni snimek, zaciname na 0
    var f = function() // funkce vykresli snimek
    {
      var frame = i++ % _timeline.length; // cislo snimku
      var parts = _timeline[frame].blit; // zmenene casti

      for (var j = 0; j < parts.length; ++j)
      {
        var part = parts[j]; // souradnice
        ctx.drawImage.apply( // vykresli zmenenou cast
          ctx,
          [img].concat(part).concat([part[2], part[3]])
        );
      }
      window.setTimeout(f, 500) // opakuj za dalsich 500ms
    };
    f(); // spust animaci
  };
  img.src = './anim_packed.png'; // nacist obrazek
</script>
\end{lstlisting}
\end{figure}
\FloatBarrier

V kódu, jak je vidět z~ukázky, se nejprve definuje vykreslující element a~načtou se metadata animace, tedy jeden ze souborů animace. V~Javascriptu se pak definuje, co se má stát při změně snímku, která je pevně definovaná každých 500ms. V~každé změně se změní části z~grafiky podle dat z~načteného souboru metadat. 


\subsection{Vývoj plnohodnotné verze}

Jakmile byl generátor animace, který spolupracoval správně s~tímto jednoduchým přehrávačem, hotov, začala práce na plnohodnotném přehrávači podle návrhu aplikace přehrávače definovaného v~kapitole \ref{section:playerdraft}. Nejprve bylo potřeba nastavit vývojové prostředí pomocí aplikace Grunt a~následně se vytvořil funkční přehrávač pro knihovnu jQuery. Implementace tříd a~komunikací mezi nimi proběhla bez větších problémů.

\subsubsection*{Implementace jako jQuery plugin}

Implementace jako jQuery pluginu byla jednoduchá. Kvůli jednoduchosti ovládání byl implementován postup, aby se instance přehrávače ukládala k~objektu. Výhodou tohoto přístupu je to, že si nemusíme ukládat instanci vytvořené animace pro pozdější kontrolu, ale můžeme ji získat opět z~daného objektu jako je ukázáno v~následující ukázce:

\begin{figure}[h]
\caption{Ukázka použití pluginu, když je uložen u elementu}
\begin{lstlisting}
  // v jedne casti aplikace
  $("#canvas").html5anim({
    src: 'animation/packed.png',
    timeline: 'animation/timeline.json'
  });

  ...

  // v jine casti aplikace
  $("#canvas").html5anim('pause');
\end{lstlisting}
\end{figure}

Celá dokumentace je podrobně popsána v~uživatelské příručce (příloha \ref{chapter:documentation}).

\section{Testování}

Javascriptový kód má kontrolovanou sémantičnost pomocí jslint\cite{jslint} tak, aby byla zajištěna přehlednost a~jednotnost kódu. Dále je otestováno veřejné API rozhraní výsledné animace pomocí nástroje Mocha\cite{mocha}.
