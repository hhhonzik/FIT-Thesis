\chapter{Implementace generátoru animace}

Tato kapitola je věnována implementaci generátoru a definovaného formátu animace z~předchozí kapitoly. Jedná se o~konzolovou aplikaci v~programovacím jazyku Python.

\section{Výběr programovacích jazyků}

Generátor bude fungovat jako aplikace běžící v~příkazovém řádku. Aplikace je určena především pro programátory, kteří jsou znalí tohoto prostředí. Pro napsání takové aplikace je možné použít řadu programovácích jazyků, z~nichž se nejvhodněji ukázal jazyk Python\cite{python}.

Python je implementován v~jazyce C\cite{c}. Charakterizuje se rychlostí, jelikož všechny kritické knihovny jsou implementovány právě v~jazyce C a~Python s~nimi velmi dobře komunikuje. Je například 3x - 5x rychlejší než jazyk PHP \cite{phpvspython}. Python není potřeba překládat do binární podoby, interpret spouští rovnou zdrojový kód, což vzhledem k~open-source povaze aplikace není problémem.  Python podporuje třídy, datové formáty a~další programátorské paradigmata objektového programování.

Dalším důvodem výběru tohoto jazyka je možnost použití knihovny Scipy\cite{Scipy}, která obsahuje několik algoritmických postupů, které použijeme při generování animace. 

Pro správný chod generátoru bude potřeba doinstalovat několik knihoven popsaných níže. Postup instalace je popsán v~Instalační příručce \ref{chapter:installation}.

\subsubsection*{Scipy}

Scipy je knihovna plná tříd a~objektů určená pro matematické výpočty, vědu nebo výzkum\cite{Scipy}. Je v~ní obsaženo velké množství funkcí včetně vykreslování grafů. Aplikace bude využívat její algoritmy nad velkým množství dat pro analýzu rozdílu jednotlivých kroků animace (kapitola \ref{section:genimpl})


\section{Postup vývoje}

Při vývoji bylo vyzkoušeno několik variant a~bylo nutno vyřešit určité problémy. Je důležité s~zmínit, že vlastní vývoj začal zejména až po definování formátu animace. V~následujícím seznamu je popsán přibližný harmonogram začátku vývoje.

\begin{itemize}
\item Získání dat z~několika obrázků, které tvoří snímky animace.
\item Nalezení rozdílů mezi těmito snímky
\item Export animace do námi vytvořeného formátu
\item Vytvoření jednoduchého přehrávače, který dokáže animaci přehrát
\end{itemize}

Po těchto úkonech bylo možné otestovat efektivitu tohoto formátu, z~čehož vznik první příklad popsaný v~sekci \ref{section:example1}. Jelikož byl formát efektivnější, pokračoval jsem v~implementaci dalších požadavků na obě části aplikace.

\subsection{Problémy}

Na úvod jsem se setkal hlavně s~problematickou kompatibilitou mezi verzemi Python 2.x a~3.x. Projekt je psaný v~jazyce Python verze 3.x, pro který není aktualizovaná spousta knihoven. To zpomalilo výběr knihovny pro porovnávání snímků animace.

Zjištění rozdílů mezi snímky a~jejich optimalizace je nakonec řešena díky knihovně Scipy.

\subsubsection*{Kompatibilita}

Vývoj probíhal na několika různých zařízeních. Při tom se objevilo množství problémů při instalaci knihoven, takže byla věnována zvláštní pozornost nastavení prostředí a~co nejjednoduššímu zprovoznění prostředí. 



\section{Realizace}
\label{section:genimpl}

Zde je popsaná adresářová struktura, která byla použita při realizaci generátoru. Byla navržena tak, aby vždy jeden soubor obsahoval jednu třídu nebo soubor podobných funkcí.


\begin{figure}[h]
    \dirtree{%
        .1 main.py\DTcomment{Spustitelný soubor generátoru}.
        .1 app.
        .2 controller.py\DTcomment{Třída Controller}.
        .2 generator.
        .3 generator.py\DTcomment{Třída Generator}.
        .2 parsers.
        .3 Interface.py\DTcomment{Rozhraní Parser}.
        .3 GifParser.py\DTcomment{Třída GifParser}.
        .3 VideoParser.py\DTcomment{Třída VideoParser}.
        .3 ImageParser.py\DTcomment{Třída ImageParser}.
        .2 export.
        .3 export.py\DTcomment{Třída Export}.
        .2 utils.
        .3 console.py\DTcomment{pomocné funkce pro výpis do konzole}.
    }
\end{figure}
\FloatBarrier


V následujících sekcích popíšu vybrané třídy, které mají zajímavou implementaci z pohledu implementace algoritmu zakódování animace.

\subsection{parsers/*}

Třídy dědící rozhraní \textbf{parsers/Interface.py} formátují vstup do námi potřebného formátu.

Pro vygenerování animačního formátu je potřeba definovat akceptované vstupní formáty. Vstup bude rozložen na~sekvenci po sobě jdoucích obrázků, které by vytvořily animaci. Tyto obrázky se budou načítat odlišně pro každý vstup. Tyto jednotlivé možnosti jsou zde rozepsány podle typu vstupu.

Vstupní metody pak načtou data do objektu \textbf{ndimage}, který poskytuje knihovna \textbf{Scipy}. Díky tomu bude možnost na snímcích animace provádět další operace.

\subsubsection*{parsers/ImageParser.py}

V této třídě se tedy pouze načtou obrázky ze složky na vstupu ve formátu \textbf{ndimage}.

Sekvence obrázků je jednoduchým vstupem. Na rozdíl od MP4 videa nebo formátu GIF zde jsou jednotlivé snímky samostatně vytvořené. Vstupem tedy bude složka s~očíslovanými obrázky. Vstupními parametry bude možné zadávat formát obrázku.

\subsubsection*{parsers/VideoParser.py}

Třída má za úkol zprocesovat data z~videa.

MP4 video je formát videa, ze kterého vytvoříme jednotlivé snímky podle zadaných vstupních parametrů - cesty k~videu a~intervalu mezi snímky. Externí knihovna ffmpeg\cite{videolib} pak vybere snímky z~videa podle intervalu. Dané snímky se pak převedou do formátu \textbf{ndimage}.

\subsubsection*{parsers/GifParser.py}

U GIF formátu bude podobný postup jako u~videa. Seznam snímků však vytváří knihovna PIL\cite{PIL} z~každého snímku z~GIF animace. Následné snímky převádí do potřebného formátu \textbf{ndimage}.

\subsection{generator/generator.py}

Pokud vstup proběhl bez problémů, je potřeba implementovat algoritmus pro zakódování animace (kapitola \ref{section:algencode}), čemuž odpovídá třída \textbf{Generator}.

V úvodu se obrázky rozdělí do dvojic - na obrázek první a~ten následující. Pak se pomocí funkce \textbf(find\_objects) najdou rozdíly mezi těmito dvěma obrázky, které jsou reprezentovány pomocí datové struktury \textbf{slice}. Toto pole se pak projde a~rozdíly, které mají k~sobě blízko se snaží sloučit na základě nějaké tolerované vzdálenosti.

Jakmile máme jsou tyto rozdíly hotové, seřadí se podle velikosti a~vytvoří se výsledný obrázek se všemy rozdíly mezi snímky. Tento obrázek a~informace o~snímcích se předají na výstup.


\subsection{export/export.py}

Celá animace je tvořena ze dvou souborů. Fragmenty animace se uloží do jednoho obrázku. Nejvhodnější je PNG formát vzhledem k~bezstrátové kompresi. Všechny rozdílné snímky animace jsou umístěné v~jednom obrázku, abychom snížili počet požadavků na~server. 

Druhým souborem jsou informace o~snímcích a~jejich použití. Tento soubor lze zakódovat do PNG obrázku díky jeho kompresi a~následné dekompresi při přehrání. Je ukládán ve formátu JSON. Jedná se o~formát, který slouží zejména pro přenos dat mezi serverem a~webovou aplikací a~tento princip bude používat i~vytvořený přehrávač. 

Výstup se řeší pomocí třídy \textbf{Export}, která nejdříve otestuje, zda je možný zápis do výstupní složky. Po vygenerování animace se pak uloží animace a~přidá se k~nim HTML šablona s~ukázkovým použitím.

\subsubsection*{Komprese výstupu}

Na výstupu se snažíme získat co nejmenší velikost kvůli datovým přenosům. Kompresi provedeme pomocí nástroje pngquant\cite{pngquant}, který optimalizuje barevné spektrum tak, byl výsledný obrázek pro lidské oko nerozeznatelný od původního nekomprimovaného souboru, ale měl přitom mnohem menší počet barev, což u~formátu PNG znamená menší velikost.

Soubor JSON můžeme zminimalizovat tak, že odstraníme přebytečné mezery a~tabulátory a~další nevýznamové znaky.

\section{Testování}

Interpret jazyka Python dokáže spustit i~kód, který například obsahuje přebytečné středníky, které jsem občas v~kódu psal ze zvyku z~jiných programovacích jazyků (C++, PHP). Pro zachování čistoty kódu je proto použit pylint\cite{pylint}, který zkontroluje veškeré nesrovnalosti v~kódu a~popřípadě vypíše chyby.

Jednotlivé části generátoru jsou testovány samostatně. Pro každou tuto část je napsaný jednotkový test, který otestuje její chování.

Testy jsou napsané pomocí frameworku Nose\cite{nose} a~vše je připraveno v~nástroji Make\cite{make}. Testy lze spustit následujícím příkazem.


\begin{lstlisting}
cd ./src/impl/generator/ && make test
\end{lstlisting}

