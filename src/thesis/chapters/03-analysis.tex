\chapter{Analýza}


V této kapitole je provedena analýza požadavků a~podle ní je navržena architektura výsledné aplikace. 


\section{Požadavky}

\subsection{Nefunkční požadavky}

Výsledný způsob animace by měl splňovat níže uvedené podmínky.

\subsubsection*{Animace v~prostředí webového prohlížeče}

Důvodem vzniku práce bylo poskytnutí formátu animace ve webovém prostředí, které bude mít díky modernějším technologiím výhodu v~efektivitě a~rychlosti. Je tedy zřejmé, že musíme dodržet základní podmínku a~to podporu ve všech majoritních prohlížečích. Vše by mělo fungovat bez použití doplňků nebo rozšíření v~prohlížečích.

\subsubsection*{Podpora pro mobilní zařízení}

Důvodem tohoto požadavku je fakt, že existující řešení mají velice malou podporu na~mobilních zařízení. Adobe Flash\cite{flash} není podporovaný vůbec, HTML5 Video\cite{html5video} je možné přehrávat pouze v~režimu na celou obrazovku, čímž se animace vytrhne z~kontextu stránky a~zobrazení je tím pádem špatné.

\subsubsection*{Ovládání animace}

Tento požadavek opět souvisí s~tím, že animací chceme stránku ozdobit v~jednom určitém momentu. Je tedy třeba poskytnout vývojářům, kteří by chtěli animovat nějakou část stránky, základní ovládací prvky.


\subsubsection*{Snadná integrovatelnost}

Pro vývojáře musí být co nejjednodušší integrace pomocí přidaných skriptů a~návodů. Animace jsou nyní uloženy jako film, tudíž je potřeba vytvořit program, který dokáže z~filmu vytvořit námi požadovaný formát.

Pokud bychom se připravovali na tento formát už od začátku, měli bychom seznam obrázků, kde každý z~nich obsahuje jeden krok animace.

\subsection{Funkční požadavky}

\subsubsection*{Vlastnosti generovacího skriptu}

Animace se bude pouštět pomocí Javascript knihovny, uživatelé stránek budou používat už předem připravenou a~vygenerovanou verzi. Pro vývojáře bude připraven generovací skript do tohoto formátu v~jazyce Python\cite{python}.

\begin{itemize}
\item Ukládání ve formátu PNG
\item Vytvoření časové osy každé animace
\item Možnost vygenerovat animaci ze sekvence snímků
\item Možnost vygenerovat animaci z~GIF obrázku
\item Možnost vygenerovat animaci z~videa s~nastavením kvality.
\end{itemize}

\subsubsection*{Vlastnosti přehrávače}

Tyto vlastnosti bude mít Javascript přehrávač animací

\begin{itemize}
  \item Možnost spustit animaci v~cyklu stále za sebou
  \item Možnost spustit animaci do konce nebo nějakého snímku 
  \item Možnost zvolit rychlost přehrávání
  \item Možnost skočit v~animaci na určitý snímek
  
\end{itemize}

\section{Případy užití}

\subsection{Vytvoření animace}

Programátor by měl mít možnost si vytvořit animaci v~tomto formátu z~dostupných zdrojů. Bude se tedy jednat o~vytvoření animace z~obrázků GIF nebo sekvence obrázků (například dostupných od grafika). Dále bude možnost generovat animaci například z~videa zachycené obrazovky.

\subsection{Přehrávání animace ve webovém prohlížeči}

Animace by měla být přehrávána ve webovém rozhraní na všech důležitých prohlížečích. Musí být zajištěna jednoduchá implementace do webové stránky pro programátora a~snadná ovladatelnost přehrávání. 


\section{Architektura projektu}

Řešení tedy rozdělíme do dvou částí. První část bude generovací skript, který animaci připraví ze vstupních formátů, druhá část se pak bude pouštět v webovém prohlížeči uživatelských počítačů. Pro každou část se budou používat trochu jiné technologie, vzhledem k~účelu použití, proto mají dva různé návrhy a~implementace.




